\section{Algorithmus} % High-level description
\todo[inline]{Textuelle Beschreibung, Grafiken}
% Idee für Layout: 
% - Gruppenelemente in Blase passen
% - festes Layout pro Hierarchiestufe berechnen
% - Layouts in Gruppen richten sich nach Ports aus Layout von Stufe darüber
% - Layouts niedrigerer Stufe beeinflussen Layouts höherer Stufe nur durch benötigten Platz
	% Stichwort: uniform scaling
	
		% bottom-up und top-down
	% (Layoutalgorithmus hierbei egtl frei wählbar, kräftebasiert aber besser in Hinsicht auf Verhalten Algorithmus mit Ankern)
	% bottom-up
		% Berechne Layouts pro Gruppe auf niederigster Stufe
		% versuche hierbei Fläche klein zu halten bzw Elemente in Kreis zu zwingen
		% Berechne die approximierte resultierende benötigte Fläche für Gruppe
	
		% Benutze appr. benötigte Fläche für Wiederholung des Schritts auf Stufe höher (bis auf letzte)

	% top-down
		% (hier noch zu entscheiden ob mit Gruppen offen oder zu)
			% (für Variante zu: schliese alle Gruppen und berechne approximierte, logarithmisch abgeschwächte Abstoßungsfaktor anhand von benötigter Größe)
		% berechne Layout auf oberster Stufe
		% Lege Ports für Gruppen auf oberster Stufe fest
		
		% berechne iterativ absteigend Layouts in Gruppen mit festgelegten Ports

\begin{algorithm}[H]
\SetAlgoLined
\Ein{Graph $G=(V,E)$ mit Gruppen $S$ } % mit Koten $V$ mit Größe, gerichteten \& gefärbten Kanten $E$, Mengen $S$ von Knoten und Elementen aus $S$
\Aus{Gruppen-hierarchisches Layout von $G$ } % mit diesen und jenen Eigenschaften, evtl Name einführen für Referenzierung
$i = $höchste Stufe einer Gruppe \;
\Solange{$i \geq 0$}{ % benötigt, dass oberste Ebene auch Gruppe
  \Fuer{Jede Gruppe $S$ auf Stufe $i$} {
	berechnete Layout der Gruppe $S$\;
	berechne benötite Fläche des Gruppenlayouts\;	
  }
  $i= i - 1$\;
}
Lege Ports für Gruppen auf Stufe 1 fest\;
$i = 1$\;
\Solange{$i \leq$ Anzahl Stufen}{ % benötigt, dass oberste Ebene auch Gruppe
  \Fuer{Jede Gruppe $S$ auf Stufe $i$} {
	berechnete Layout der Gruppe $S$ unter Berücksichtigung der Ports\;
	Lege Ports für Gruppen auf Stufe $i+1$ fest;
  }
  $i= i + 1$\;
}
\caption{Layoutalgorithmus}
\end{algorithm}

% DEFINITION Anker
% Pseudoknoten, wird nicht gerendert
% hat Kanten zu allen Knoten, welche er Ankert
% optimale Kantenlänge = 0
% Federkraft Fallabhängig und jeweils zu spezifizieren

\subsection{Layout in Gruppen in Abhängigkeit von Ports}
% Problem: Layout eines Graphens (mit Labelknoten) innerhalb eines Kreises mit Ports für ausgehende kanten
% Äquivalent: Layout eines Graphens mit punktförmigen Knoten auf Kreis und gelabelten Knoten innerhalb des Kreises
\todo[inline]{Problemreduktion beschreiben, Lösungsvorschlag}

% Bemerkung: Layout in einer Gruppe ist von Layouts in anderen Stufen und Gruppen unabhängig (außer durch Ports)

% Genau zu spezifizieren:
% - wie wird es modelliert: Ports als feste Knoten? nur Reihenfolge festlegen und flexibel auf gewissem Radius? ..
	% Ports bleiben fest. Argument: Sowieso Winkel an Ports, nich wesentliche Winkel-Änderung erwartet bei Veränderung

% - Anziehung zu Mittelpunkt
	% ja, mit Kanter
	% schwache Feder mit Faktor $\alpha$

% - Anfangslayout: 
\subsubsection{Anfangslayout}
	% a) Knoten mit aus Gruppe ausgehende eingehende Kanten direkt neben Ports positionieren, dann breitensuche oder so weiter
	% b) random
	% c) Layout von Größenberechnung spiegeln (nicht, vertikal, horizontal, beides) und Längen der Port-Knoten-Kanten summieren

% - Größe des Kreises definiert durch Radius R
\subsubsection{Bestimmen des Kreisradius}
	% Gegeben: Berechnete Größe R' in bottom-up Verfahren
	% Problem: berechnete Größe ohne Ports, kann zu klein sein, da Platz für Kanten zu Ports geschaffen werden muss (darf nicht durch Knoten)
	% Ziel: finde Faktor $\beta$ sodass $R = \beta * R'$ groß genug            ( es gillt immer $R \geq R'$)
	
	% Variante induvidualisiertes $\beta$ für jede Gruppe - anfangs 1. Startlayout  und später 2. für jeden Zustand 
	  %________________________________________________________________________________________________________________%
	 	% Algorithmus zum finden von $\beta$
			% Idee: berechne Layout, teste ob Radius verkleinert werden kann, erhöhe bei bedarf Anziehung zu Mitte ($\alpha$)
			% Abbruchbedingungen: - Änderung in Größe nach erhöhen von Anziehung zu Mitte zu klein
			%			         - Anziehung zu Mitte wurde schon stark erhöht und man ist schon nahe an R' 
								% $\epsilon = f(\alpha)$, wenn $R \leq epsilon * R' $   Abbruch
	  %
		%   0. $\epsilon = 1$   (Abbruchvariable)
		%   1. Setze $R = \beta' * R'$ mit $\beta'$ groß genug,  s
		%   2. Wenn $\beta' < \epsilon$ gehe zu 8.
		%   3. Berechne Layout mit Anfangslayout in erster Runde oder Layout von letzter Runde (und natürlich Kanten zu Ports)
		%   4. Wenn $R$ verkleinert werden kann wähle neues $\beta'$ und gehe zu 1.
		%   5. Erhöhe Mittelankerfederstärke $alpha$, erhöhe $\epsilon$
		%   6. Berechne Layout
		%   7. Wenn mögliche Änderung  $ > x\%$, wähle neues $\beta'$ und gehe zu 1.
		%   8. $\beta = \beta'$
	 %_________________________________________________________________________________________________________________%		


	% Variante festes $\beta$ und mittels Berechnung von erwarteten Größen
	%_________________________________________________________________________________________________________________%
		% Idee: Berechne ausreichende Größe anhand von Größen der Kinder
		
		% Für Kindruppe $G_i$ ist $x_i = ((Radius in Zustand von G_i) - (max Radius von G_i) ) \cdot \xi$, wobei $0 < \xi < 1$
		% $R = R' + x_i  + \delta$
	%_________________________________________________________________________________________________________________%
							

	% Anmerkung: Da in Praxis (Argumentkarten) die Gruppentiefe nicht hoch ist und in Gruppe nicht viele Gruppen sind, 
			% können hier auch die Größen für alle Fälle berechnet werden

% Postprocessing: Kantenklätung

\subsection{Layout-Anpassung beim Öffnen oder Schließen einer Gruppe}
% Removing node intersections when changing the size of a node while keeping the mental map
% Problemabstraktion: Auflösen von Knotenlabelüberschneidung nach Größenveränderung eines Knotenlabes unter Beibehaltung der mentalen Karte
\todo[inline]{Problemreduktion beschreiben, Lösungsvorschlag}

% Idee für Verhalten: Anker
% was passiert, wenn Gruppe geöffnet oder geschlossen wird?
% - jedes Element auf oberster Ebene bekommt festen aber schwachen Anker an Startposition von Anfangslayout um immer dazu ähnlich zu bleiben
% - jedes Element auf gleicher Ebene wie veränderte Gruppe bekommt einen Anker an aktueller Position
	% geöffnete Gruppe stoßt alle Elemente ab, sodass sie genug Platz hat.  Rest auch kräftebasiert um Ähnlichkeit zu Layout davor zu bewahren
	% geschlossene Gruppe stoßt nun Elemente weniger ab. Rest wieder kräftebasiert um Ähnlichkeit zu Layout davor zu bewahren
% - Ist geänderte Gruppe nicht auf oberster Ebene, so wiederholt sich der Prozess iterativ nach oben, da übergeordnete Gruppen auch mehr Platz brauen oder freigeben

% Genau zu spezifizieren:
% - Wann werden Anker gesetzt, wie lange bestehen sie
	% Grundanker, Stärke muss man testen, evtl so stark wie eine Kante $\cdot \gamma$ mit $\gamma \in [1, irgendwas]$
	% Abhängig von Kräftealgorithmus, wenn dort Masse von Gruppen nicht mit einbezogen wird, dann für größere Gruppen, stärkere Anker
	% ((( Erweiterungsmöglichkeit: Anker von jedem Schritt speichern, jedoch pro Schritt abschwächen)))
% - Verhältnis Gruppengrößen und Kräfte für Abstoßungen/Anziehungen
	% Variante zu entscheiden: a) Größe proportional, dann linear/logarithmisch
	%				  b) Abstoßung proportional, dann logarithmisch
% - Verhalten von Ports bei Layoutänderung
	% bleiben fest. keine großen Änderungen zu erwarten.

% Bei zu starker Veränderung also zu viel Kraft auf Ankern: verwerfe Grundanker und setze für diese Runde neue
% evtl ebenso für Ports, wenn Knick zu groß


% ------- Vorgehen für Parameterwahl vorschlagen
