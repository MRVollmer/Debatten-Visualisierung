\section{Evaluation und Vergleich zu anderen Lösungsansätzen}
\subsection{Hierarchisches Layout}
\todo[inline]{Beschreibung hierarchische Layout Variante}
\subsection{Gänzlich kräftebasiertes Layout} % TODO Besseren Namen finden
\todo[inline]{Beschreibung kräftebasierte Layout Variante}
% zu vergleichen mit hierarchischem Layout und alles kräftebasiert

\subsection{Vergleich}
\todo[inline]{Vergleich, Evaluation schreiben}
% gesammelte Argumente
	% KB alles kräftebasiert, HS hierarchisch/''Stufen''-Layout
% + Gruppen vorallem semantisch organisiert, weniger strukturell als bei KB
% + Änderungskonstanz der Gruppen (innerhalb von Gruppe) als bei KB
% - komplexer als KB
% + Gruppen überschneiden sich nicht wie bzw einfach zu getrennt zu halten als bei KB 
% + HS ist nur ohne Zykel gut umsetzbar
% + komptakter als HS
% - Struktur evtl nicht so sichtbar wie bei HS aber auch wie bei KB
% + iterativ anwendbar
% + schöner :)
% o Gruppen im Vordergrund der Darstellung durch Ports, nicht Knoten
% o Kantenrounting nicht unbedingt trivial
% 
