\chapter{Evaluation und Vergleich zu anderen Lösungsansätzen}
\label{chapter:vgl}

Bei der Entwicklung unserer Lösung sind noch andere Ansätze entstanden, welche nach Vergleich mit unserem Ansatz (der in \autoref{chapter:algo} beschrieben wird), aber nicht weiter ausgearbeitet oder ausgewertet wurden. Der Vollständigkeit wegen, stellen wir nun noch kurz einen alternativen kräftebasierten Ansatz (siehe \autoref{UniformK-Ansatz}) und einen hierarchischen Ansatz (siehe \autoref{Hierarch-Ansatz}) vor. Ein kurzer Vergleich wird in \autoref{Ansatz-Vergleich} angestellt.

\section{Uniform kräftebasiertes Layout}
\label{UniformK-Ansatz}
Dieser Lösungsansatz stellt eine einfache Erweiterung zu vorhandenen kräftebasierten Algorithmen dar: Knoten können durch zusammenfassende Pseudoknoten, in diesem Fall Gruppenknoten, repräsentiert werden und zusätzlich werden Anker verwendet.

Konkret bedeutet dies, dass Knoten zunächst nur durch ihre Gruppenknoten dargestellt werden und dann mit einem einfachen kräftebasierten Verfahren ein Layout bestimmt wird. Hierbei wird die abstoßende Kraft der Pseudoknoten bzw. die Anziehungskraft zusammengelegter Kanten logarithmisch mit der Anzahl zusammengefasster Elemente skaliert.

Wenn nun die Elemente einer oder mehrerer Gruppen konkret dargestellt werden sollen, werden zunächst alle Gruppenknoten, ähnlich wie bei der finalen Lösung, verankert. 
Zusätzlich werden die Gruppenknoten, deren Elemente dargestellt werden sollen, aus dem Layout entfernt und stattdessen deren Elemente eingefügt. Diese können ebenfalls Gruppenknoten für innere Gruppen sein. Schließlich wird nun noch einmal der kräftebasierte Algorithmus verwendet mit der Besonderheit, dass jeder Knoten eine zusätzliche Anziehung zu seinem Anker bzw. dem vom übergeordneten Gruppenknoten geerbten Anker hat.

Dieser Ablauf wird in \autoref{UniformK-Skizze} skizziert.

\begin{figure}[h!]
\begin{center}
	\includegraphics[width=\textwidth]{Pics/uniform_kb.pdf}
	\caption{Skizze eines Layouts durch den uniform kräftebasierten Ansatz. Links zunächst das Layout mit alle Gruppen geschlossen. In der Mitte das Layout nach Öffnen der hellblauen Gruppe. Rechts das Layout mit hellblauer und grüner Gruppe geöffnet. Kreuze markieren hierbei die Anker der Gruppen.}
	\label{UniformK-Skizze}
\end{center}
\end{figure}

% zu vergleichen mit hierarchischem Layout und alles kräftebasiert

\section{Hierarchisches Layout}
\label{Hierarch-Ansatz}
Häufig auftretende Eigenschaften in Argumentkarten, wie beispielsweise Zyklenfreiheit, legen andere Graphenlayouts, insbesondere hierarchische Layouts, nahe.

Wenn wir die Richtung der Kanten, entgegen der bisherigen Ansätze, für das Layout beachten wollen, wird es dadurch möglich bei einem hierarchischen Layout alle eingehenden Kanten eines Knotens an der Oberseite des Knotens und alle ausgehenden Kanten an der Unterseite des Knotens zu zeichnen. 
Dadurch wird ebenfalls das \glqq Einpacken\grqq\ der Knoten in eine Box zur Repräsentation bzw. Ordnung der Gruppe einfacher, da auch die Box der Gruppe die ein- und ausgehenden Kanten entsprechend ordnen kann ohne die Komplexität des Layouts innerhalb der Box zu erhöhen. Dadurch kann das innere und äußere Layout von Gruppen getrennt berechnen werden.

Ein grobe Skizze zu einem solchen Layout ist in \autoref{Hierarch-Skizze} dargestellt.

\begin{figure}[h!]
\begin{center}
	\includegraphics[width=\textwidth]{Pics/hierarchisch.pdf}
	\caption{Skizze eines Layouts durch den hierarchischen Ansatz. Links das Gesamtlayout, Mitte und Rechts jeweils das Layout der jeweiligen Gruppe.}
	\label{Hierarch-Skizze}
\end{center}
\end{figure}

\section{Vergleich} 
%\todo[inline]{Kriterien \& Übersicht} % erfüllt oder?
\label{Ansatz-Vergleich}
Im Folgenden vergleichen wir die beiden skizzierten Ansätze mit unserem Lösungsansatz. Wir vergleichen Ästhetik-Kriterien (Layoutfokus und Änderungskonstanz), Einfluss auf die mentale Karte (Änderungskonstanz), sowie die Praktikabilität (Algrithmuskomplexität und Graphenanforderungen).

Da keiner der Ansätze konkret implementiert wurde und die alternativen Ansätze auch nur stückweise ausgearbeitet wurden, besteht der Vergleich dieser Ansätze größtenteils aus Abschätzungen und nicht aus tatsächlichen Messungen.

%\newpage

Es werden folgende Abkürzungen verwendet:

\textbf{KBK}: Unser Lösungsansatz, der kräftebasierte Algorithmus mit Kapselung

\textbf{HL}: Der hierarchische Layout-Ansatz

\textbf{UKB}: Der uniform kräftebasierte Layout-Algorithmus

\myparagraph{Fokus des Layouts}
Bei KBK und HL sind die Layouts der einzelnen Gruppenebenen größtenteils unabhängig. Daher liegt der Fokus der Darstellung eher auf der Semantik der Gruppen, d.h. das Layout stellt besonders heraus, zu welcher Gruppe ein Knoten gehört.

Das durch UKB erzeugte Layout basiert dagegen größtenteils auf den Knotenzusammenhängen. Durch die Anker bleiben Gruppenelemente zwar größtenteils zusammen, aber das Layout wird dennoch über die Kanten der Knoten in und zwischen Gruppen festgelegt. Bei entsprechenden Zusammenhängen können Gruppen auch deformiert oder getrennt werden (siehe \autoref{Uniform-Trenn-Skizze}).

\begin{figure}[h]
\begin{center}
	\includegraphics[width=\textwidth]{Pics/uniform_trenn.pdf}
	\caption{Beispielhafter Ablauf der zur Trennung einer Gruppe führt: Durch ungünstige Kantenzusammenhänge und einen entscheidenden Größenunterschied drängt sich die weiße Gruppe zwischen die Knoten der schwarzen Gruppe.}
	\label{Uniform-Trenn-Skizze}
\end{center}
\end{figure}


\myparagraph{Kompaktheit}
Sowohl bei KBK als auch bei HL verliert man durch die Kapselung von Gruppen viel Kompaktheit, da hierbei oft Räume in den Gruppen entstehen, die nicht von anderen Gruppen und Knoten genutzt werden können. Das Gesamtlayout bei KBK ist allerdings bei allen Graphen eher zentriert und rund, da es kräftebasiert bestimmt wird, während HL bei unausgeglichenen Graphen zu besonders hohen bzw. breiten Graphen führt.

Das Layout des UKB-Ansatzes führt zu ähnlich kompakten Graphen wie herkömmliche kräftebasierte Algorithmen, da die einzigen zusätzlichen Kräfte, die Anker, an den Positionen der Gruppenknoten gesetzt werden, also in der Hüllkurve des Graphen liegen, und ausschließlich anziehend wirken.

\myparagraph{Änderungskonstanz}
Durch die Unabhängigkeit interner und externer Gruppenlayouts bei KBK und HL, sind die Änderungen beim Öffnen und Schließen von Gruppen eher gering: Die Layouts in den Gruppen sind unabhängig davon, ob andere Gruppen geöffnet oder geschlossen sind, und die relativen Positionen der Gruppen zueinander ändern sich selten.

Da bei UKB die Gruppen nicht \glqq abgeschirmt\grqq\ sind, ist das Layout geöffneter Gruppen sehr stark vom Zustand anderer Gruppen abhängig. Es ist zu erwarten, dass die relativen Position durch die Anker erhalten bleiben, aber kleine Gruppen durch große Gruppen deformiert oder verdrängt werden.

\myparagraph{Komplexität des Algorithmus}
Wie bereits in \autoref{chapter:algo} dargestellt, ist der für KBK nötige Algorithmus recht komplex und enthält einige Parameter, die noch bestimmt werden müssen.

Der für HL benötigt Algorithmus besitzt etwas weniger Komplexität als KBK, da die Ports für Kanten stets fest sind, wodurch die Größe von geöffneten Gruppen einmalig \glqq Bottom-Up \grqq\
und ohne Approximationen sowie Korrekturen bestimmt werden kann. Allerdings besitzt der zugrundeliegende Algorithmus für hierarchisches Layouts eine höhere Komplexität als kräftebasierende Verfahren.

Der UKB-Algorithmus stellt schließlich eine algorithmisch weniger komplexe Erweiterung von einfachen kräftebasierten Verfahren dar, was vermutlich zu der geringsten Komplexität unter den hier verglichenen Ansätzen führt.

\myparagraph{Anforderung an Graphen}
Während alle hier vorgestellten Ansätze davon ausgehen, dass jeder Knoten und jede Gruppe maximal einer übergeordneten Gruppe zugehört, werden nur bei HL tatsächliche Anforderungen an die eigentliche Graphenstruktur (Zyklenfreiheit) gestellt, um problemlos zu funktionieren.

\section*{Fazit}
Wie man an den Einzelpunkten erkennt, vereint der Ansatz für den wir uns entschieden haben viele der positiven Eigenschaften der beiden anderen Ansätze oder stellt einen Kompromiss zwischen den beiden dar.
%\todo[inline]{Allgemeine Auswertung des Ansatzes/Verfahren schreiben?}
% gesammelte Argumente
	% KB alles kräftebasiert, HS hierarchisch/''Stufen''-Layout
% + Gruppen vorallem semantisch organisiert, weniger strukturell als bei KB
% + Änderungskonstanz der Gruppen (innerhalb von Gruppe) als bei KB
% - komplexer als KB
% + Gruppen überschneiden sich nicht wie bzw einfach zu getrennt zu halten als bei KB 
% + HS ist nur ohne Zykel gut umsetzbar
% + komptakter als HS
% - Struktur evtl nicht so sichtbar wie bei HS aber auch wie bei KB
% + iterativ anwendbar
% + schöner :)
% o Gruppen im Vordergrund der Darstellung durch Ports, nicht Knoten
% o Kantenrounting nicht unbedingt trivial
% 
