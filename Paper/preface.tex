\section{Einleitung}

%\begin{center}
%\vspace{-9cm}
%\includegraphics[width = 0.3\linewidth]{Pics/KITLogo.jpg}
%\vspace{6.5cm}
%\end{center}


%
\todo[inline]{Motivation schreiben}
% Was sind Argumentkarten? Wer verwendet sie und wofür?

% Darstellung von Argumentkarten
% - Wie werden sie bis jetzt dargestellt?
% - Wieso will man sie darstellen?

% Spezielle Anforderungen
% - Wieso Gruppen?
% - Wieso Layout beibehalten? -> Mentale Karte

% Wie ist diese Ausarbeitung aufgebaut?
\subsection{Gliederung}
% 1. Problemstellung und -definition
% 2. Lösungsvorschlag
% 3. Detaillierte Erklärung des Algorithmus
% 4. Vergleich zu anderen Lösungsansätzen

\section{Problemstellung}
\todo[inline]{Problem definieren, Layoutproblem und Subprobleme, Layoutanspassungsproblem}
Wir können für diese Problem ein Layoutproblem definieren.

\begin{description}
\item[Eingabe] Gegeben ein Graph $G=(V,E)$ mit Knoten $V = \{ \text{achsenparallele Rechtecke}\}$ und  Kanten $E \subseteq V \times V$,
sowie einen Gruppenzugehörigkeitsbaum 

\item[Zeichenkonvetionen]
	\begin{itemize}
	\item Geradlinige Kanten
	\item $\forall x,y\in V, x \neq y: x \cap y = \emptyset$, also Knoten paarweise disjunkt
	\item $\forall x \in V, (y,z) \in E, y\neq x\neq z: x \cap (y,z) = \emptyset$, also Kanten schneiden nur inzidente Knoten
	\end{itemize}

\item[Ästhetikkriterien]

\item[Lokale Nebenbedingungen]

\end{description}

% Layoutproblem
% - Gegeben/Eingabe und Abstraktion
	% Graph mit Knoten $V = \{ \text{achsenparallele Rechtecke}\}$, und  Kanten $E \subseteq V \times V$
	% sowie transitive, rechtseindeutige hierarchische Struktur von Gruppen von Knoten und Gruppen
	% Zustand, welche Gruppen offen/zu
	% Anfangslayout (!?)
% - Vorgaben an Layout: 
	% Zeichenkonventionen, hard constraints:
	% - Knoten und Knoten überschneiden sich nicht (jedoch dürfen Kanten Gruppen schneiden, wenn sie zu Element der Gruppe gehören)
	% - Gruppen umschließen alle ihre Elemente (welche zusammengeklappt halt in einem Punkt zusammenfallen)
	% Ästhetikkritieren
	% - Kreuzungsminimierung
	% - gleichmäßige Kantenlängen (?)
	% - Wendepunktminimierung von Kanten
	% - Struktur hervorheben in Gruppen (?)
	% Lokale Nebenbedingungen
	% - Nummerierung von Nachbarknoten im Layout repräsentieren
	

% Verhalten beim Öffnen oder Schließen von Gruppen
% - Vorgaben an Verhalten: 
	% Ästhetikkritieren
	% - relative Positionen von Knoten und Gruppen verändern sich nur wenig
	% - Stuktur vor Zustandsänderung bleibt ähnlich und wiedererkennbar
	
% Gewählter Zeichenstil
% - Argumente als Rechtecke (mit fester Breite?)
% - Gruppen als Kreise
% - Kanten als ``glatte'' Kurven (bis zu welchem Grad?)
% - Kanten in Gruppen rein/aus Gruppe raus zu/von inneren Elementen nur über Ports







%--END--END--END--END--END--END--END--END--END--END--END--

%--END--END--END--END--END--END--END--END--END--END--END--