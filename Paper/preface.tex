%\begin{center}
%\vspace{-9cm}
%\includegraphics[width = 0.3\linewidth]{Pics/KITLogo.jpg}
%\vspace{6.5cm}
%\end{center}
%

\chapter{Motivation}
Diese Arbeit beschäftigt sich mit Argumentkarten, manchmal auch Argumentationskarten genannt. Diese dienen als graphische Darstellung von Debatten. 
Auf einer solchen Karte werden die Thesen und Argumente der Debatte als einzelstehende Elemente herausgestellt. 
Um ihre Zusammenhänge zu verstehen, ist es auch nötig ihre Beziehungen, welche von unterstützenden oder angreifender Natur sein können, ebenfalls zu visualisieren.
Eine Argumentkarte ist also die Darstellung einer Debatte in Form eines Graphen.
%Debatten werden in der Regel in sogenannten Argumentkarten, manchmal auch Argumentationskarten, dargestellt. 
%Auf einer solchen Karte werden die Thesen und Schlüsse der Debatte in einzelne Sätze aufgeschlossen. Diese Sätze werden dann in einem Graphen abgebildet. 

\todo[inline]{Beispiel einfügen}

Die Argumentkarten sind vor allem ein Werkzeug zum besseren Verständnis von der Debatte. 
Über die Visualisierung der Elemente und ihrere Beziehungen hinaus, existieren weitere Möglichkeiten um das Verstehen der Debatte zu verbessern und zu erleichtern.
So können beispielsweise thematisch verwandte Knoten gruppiert werden. 
Auch durch die Option, Gruppen zunächste geschlossen zu visualisieren und Details zu verbergen, könnnen Gruppen einen schnelleren Überblick über die Debatte ermöglichen.
Das Hervorheben einzelner Agrumentationsstrukturen durch die Wahl des Layouts ist eine weiterer wichtiger Aspekt bei der Visualisierung von Debatten.
%Dabei können verschiedene Aspekte der Debatte in den Vordergrund gerückt werden um diesen besser zu verstehen. 
%Beispielsweise können thematisch verwandte Knoten Gruppiert werden um einen besseren überblick über die Debatte zu ermöglichen 
%oder es können einzelne Arugmentationsstränge besonders hervorgehoben werden.

Damit sich die von einer Argumentkarte bereitgestellten Informationen sinnvoll verwenden lassen muss die Argumentkarte folglich auf eine geeignete Weise dargestellt werden.
Es existieren bereits einige Werkzeuge mit denen Argumentkarte angelegt  und ein Layout automatischzu erzeugt werden kann. 
Allerdings müssen die resultierenden Layouts der Argumentkarten oftmals von Hand nachbearbeitet werden, um eine befriedigende Darstellung der Strukturen zu erhalten.
Darüber hinaus ist eine Unterstützung von Gruppen nicht immer gegeben.

In dieser Arbeit geht es um die automatisierte Darstellung von Argumentkarten mit Gruppen, sowie die Interaktion mit diesen Gruppen. 
%und damit um eine Möglichkeit die Argumentkarte zu vereinfachen um dem Benutzer einen schnellen und guten Überblick über die Debatte zu ermöglichen.
Es geht also um die Frage, wie sich Gruppen und Gruppenhierarchien den Anforderungen an Argumentkarten entsprechend visualisieren lassen 
und wie das Layout auf das Öffnen und Schließen von Gruppen reagiert. 
Im nächsten Abschnitt \ref{chapter:layoutproblem}  präzisieren und abstrahieren wir diese Problemstellung. 
In Abschnitt \ref{chapter:algo} präsentieren wir unseren Lösungsansatz, welchen wir dann in \ref{chapter:vgl} mit alternativen Ansätzen vergleichen und bewerten.

% Was sind Argumentkarten? Wer verwendet sie und wofür?

% Darstellung von Argumentkarten
% - Wie werden sie bis jetzt dargestellt?
% - Wieso will man sie darstellen?

% Spezielle Anforderungen
% - Wieso Gruppen?
% - Wieso Layout beibehalten? -> Mentale Karte

% Wie ist diese Ausarbeitung aufgebaut?
%\subsection{Gliederung}
% 1. Problemstellung und -definition
% 2. Lösungsvorschlag
% 3. Detaillierte Erklärung des Algorithmus
% 4. Vergleich zu anderen Lösungsansätzen

\chapter{Layoutprobelm}
\label{chapter:layoutproblem}
Das Zeichnen eines Graphen ist ein algorithmisches Problem, dessen Anforderungen sich hauptsächlich an ästhetischen Anforderungen des Nutzers und gegebenen Laufzeitbeschränkungen orientiert. Dieses Problem wird im folgenden als Layoutproblem beschrieben.

\section{Formale Beschreibung}
Die Darstellung einer Argumentkarte die durch einen Graphen Beschrieben wird lässt sich allgemein beschreiben als folgendes Problem.

\begin{description}
	\item[Eingabe] \hfill \\[0.3\baselineskip]
			Graph $D=(V,E)$ mit Knoten $V$ und Kanten $E \subseteq V \times V$ \hfill \\
			% = \{ \text{achsenparallele Rechtecke}\}
			sowie einen Gruppenzugehörigkeitsbaum 
	
	
	\item[Ausgabe] \hfill \\[0.3\baselineskip]
		Zeichnung mit:
		\begin{itemize}
		\item Geradlinige Kanten
		\item $\forall x,y\in V, x \neq y: x \cap y = \emptyset$, also Knoten paarweise disjunkt
		\item $\forall x \in V, (y,z) \in E, y\neq x\neq z: x \cap (y,z) = \emptyset$, also Kanten schneiden nur inzidente Knoten
	\end{itemize}
\end{description}

\section{Layoutanforderungen}
Da es sich bei Argumentkarten um Graphen handelt ist die Darstellung einer Argumentkarte auch ein Layoutproblem.
Anders als bei gewöhnlichen Graphen muss allerdings auch die Semantik des Graphen in die Zeichnung einfließen.
%In unserem Fall beschränkt sich die semantische Betrachtung auf die Beziehungen von gruppierten Sätze und 

Die Anforderungen an ein Layoutalgorithmus für die Darstellung von Argumentkarten mit Gruppen lassen sich demzufolge in allgemeine und in gruppensensitive Anforderungen Teilen.

\section{Allgemeine Layoutanforderungen}
\label{sec:generalreq}
Allgemeine Layoutanforderungen sind solche bei denen die Semantik der Argumentkarte keine Rolle spielt.
Diese Anforderungen sind im einzelnen:

\begin{itemize}
\item {\normalfont \bfseries{Überschneidungsfreiheit von Knoten:}} \newline Knoten dürfen sich nicht überschneiden.
\item {\normalfont \bfseries{Überschneidungsvermeidung von Kanten:}}  \newline Kantenüberschneidungen sollten vermieden werden.
\item {\normalfont \bfseries{Größenminimierung:}} \newline Die Gesamtgröße des Layouts sollte möglichst klein sein.
\end{itemize}

\section{Gruppensensitive Anforderungen}
\label{sec:groupreq}
Die gruppensensitive Anforderungen sind solche, die direkt von der Gruppierung von Aussagen oder Argumenten abhängig.

\subsection{Verständlichkeit}
Derzeitig werden sowohl die Gruppen in Argumentkarten von Hand erstellt als auch deren Darstellung im Layout der Argumentkarte. Die Gruppierung dient dabei maßgeblich dem Verständnis der Argumentkarte und beruht und hilft es die Argumentkarte leichter zu verstehen unter dem Aspekt der Begrenztheit des menschlichen Kurzzeitgedächtnisses (siehe \cite{miller1956magical, BBS:84441}).

Ein entsprechender Layoutalgorithmus muss in der Lage sein die Gruppen so darzustellen, das die Beziehung der Gruppe zum Rest der Argumentkarte verständlich bleibt.

\subsection{Nachvollziehbarkeit}
Zusätzlich zur Verständlichkeit muss das Layout auch Nachvollziehbar sein. Zur Nachvollziehbarkeit gehört sowohl die Nachvollziehbarkeit einer statischen Abbildung der Argumentkarte, wie auch die Nachvollziehbarkeit von Änderungen an dieser Abbildung.

Damit eine statische Abbildung nachvollziehbar ist muss diese die von der Gruppe verschleierten Informationen, wie Größe oder Relation der Gruppenmitglieder zum Rest der Argumentkarte durch geeignete Darstellung der Gruppe angedeutet werden.

Eine Notwendige Änderung an einer Abbildung von Argumentkarten die Gruppen enthält ist das Auf- und Zuklappen der Gruppen um eine detailliertere Ansicht bzw. einen besseren Überblick zu ermöglichen.

Nachvollziehbar ist eine daraus resultierende Änderung des Graphen dann, wenn sie in kleinen Schritten erfolgt. Dies bedeutet in unserem Fall, dass das Auf- und Zuklappen von Gruppen den Graphen die Gesamtstruktur der Zeichnung nur in relativ kleinen Schritten verändern soll.

Zur Nachvollziehbarkeit gehört aber auch das das Layout nach wiederholten Auf- und Zuklappzyklen immer wieder zu einem dem Anfangslayout ähnlichen Layout zurückkehrt.
\todo[inline]{Mentale Karte}

%\todo[inline]{Problem definieren, Layoutproblem und Subprobleme, Layoutanspassungsproblem} Wir können für diese Problem ein Layoutproblem definieren.

%\begin{description}
%\item[Eingabe] Gegeben ein Graph $G=(V,E)$ mit Knoten $V = \{ \text{achsenparallele Rechtecke}\}$ und  Kanten $E \subseteq V \times V$,
%sowie einen Gruppenzugehörigkeitsbaum 
%
%\item[Zeichenkonvetionen]
%	\begin{itemize}
%	\item Geradlinige Kanten
%	\item $\forall x,y\in V, x \neq y: x \cap y = \emptyset$, also Knoten paarweise disjunkt
%	\item $\forall x \in V, (y,z) \in E, y\neq x\neq z: x \cap (y,z) = \emptyset$, also Kanten schneiden nur inzidente Knoten
%	\end{itemize}
%
%\item[Ästhetikkriterien]
%
%\item[Lokale Nebenbedingungen]
%
%\end{description}
%
%\section{Lösungsansätze}

% Layoutproblem
% - Gegeben/Eingabe und Abstraktion
	% Graph mit Knoten $V = \{ \text{achsenparallele Rechtecke}\}$, und  Kanten $E \subseteq V \times V$
	% sowie transitive, rechtseindeutige hierarchische Struktur von Gruppen von Knoten und Gruppen
	% Zustand, welche Gruppen offen/zu
	% Anfangslayout (!?)
% - Vorgaben an Layout: 
	% Zeichenkonventionen, hard constraints:
	% - Knoten und Knoten überschneiden sich nicht (jedoch dürfen Kanten Gruppen schneiden, wenn sie zu Element der Gruppe gehören)
	% - Gruppen umschließen alle ihre Elemente (welche zusammengeklappt halt in einem Punkt zusammenfallen)
	% Ästhetikkritieren
	% - Kreuzungsminimierung
	% - gleichmäßige Kantenlängen (?)
	% - Wendepunktminimierung von Kanten
	% - Struktur hervorheben in Gruppen (?)
	% Lokale Nebenbedingungen
	% - Nummerierung von Nachbarknoten im Layout repräsentieren
	

% Verhalten beim Öffnen oder Schließen von Gruppen
% - Vorgaben an Verhalten: 
	% Ästhetikkritieren
	% - relative Positionen von Knoten und Gruppen verändern sich nur wenig
	% - Stuktur vor Zustandsänderung bleibt ähnlich und wiedererkennbar
	
% Gewählter Zeichenstil
% - Argumente als Rechtecke (mit fester Breite?)
% - Gruppen als Kreise
% - Kanten als ``glatte'' Kurven (bis zu welchem Grad?)
% - Kanten in Gruppen rein/aus Gruppe raus zu/von inneren Elementen nur über Ports
