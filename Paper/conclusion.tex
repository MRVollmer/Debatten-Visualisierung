\chapter{Zusammenfassung und Ausblick}
\label{ch:zsf}
In dieser Arbeit haben wir mit dem Layoutproblem für Argumentkarten beschäftigt, insbesondere bezüglich vordefinierten Gruppen von Argumenten, d.h. Gruppenzugehörigkeiten sind ein Teil des zu zeichnenden Graphen im Gegensatz zu Gruppen, die als Teil des Layouts bestimmt werden (wie bspw. Graph-Clustering).

Unser Lösungsansatz stellt ein Konstrukt dar, dass aufbauend auf gebräuchlichen kräftebasierte Algorithmen, Layouts erstellt, die vordefinierte Gruppen herausstellen. Zusätzlich erlaubt der vorgestellte Ansatz das Anzeigen von Gruppen in geöffneten oder geschlossenen Zuständen. Besonderer Fokus wurde zusätzlich auf das Ästhetikkriterium gelegt, dass das Layout sich durch Öffnen und Schließen der Gruppen nur gering verändert.
% Die Wahl des Layouts und der Algorithmen erlaubt es, ein einfaches Vorgehen für den Wechsel von Layouts bei einer Interaktion. 
% ---
% Kann mit deinem Satz leider immernoch nicht so viel anfangen, was genau soll er ersetzen?

Eine konkrete Implementierung und entsprechende Auswertung wurde in dieser Arbeit nicht durchgeführt.
Bei einer groben Abschätzung scheint unser Algorithmus keine Operation zu beinhalten, die asymptotisch aufwändiger ist als die quadratische Laufzeit eines gewöhnlichen kräftebasierenden Algorithmus.
%\todo[inline]{Citation needed?}

Zukünftige Arbeiten können sich mit der in dieser Arbeit unterlassenen Implementation und Auswertung beschäftigen. Ein Teil einer solchen Implementierung sollte es sein, die Parameter, welche bei der Beschreibung des Algorithmus bewusste nicht festgelegt wurden, konkret abzuschätzen.
% Außerdem Vorschläge: Öfters hoch und runter bei Layoutberechnung. Ports nicht ganz fest. Kanten abrunden

Alternativ sind auch Ausarbeitungen und Untersuchungen der alternativen Ansätze, die diese Arbeit nur grob umrissen hat, möglich.

