\chapter{Zusammenfassung und Ausblick}
In dieser Arbeit haben wir mit dem Layoutproblem für Argumentkarten beschäftigt, insbesondere bezüglich %semantischen %> wieso semantisch?
Gruppen von Argumenten, 
d.h. Gruppenzugehörigkeiten ein Teil des zu zeichnenden Graphen sind im Gegensatz zu Gruppen, die als Teil des Layouts bestimmt werden (wie bspw. Graph-Clustering).

Für dieses Problem haben wir drei Lösungsansätze vorgestellt und verglichen. 
Einen dieser drei Ansätze %>welchen? Name
haben wir weiter ausgearbeitet und das Vorgehen detailliert beschrieben:

Dieser Ansatz % >auch hier Name
verwendet und erweitert bisherige kräftebasierte Algorithmen um vordefinierte Gruppen beim Layout zu respektieren. 
%> ist es nicht viel mehr ein zusammenspiel von mehreren algorithmen um gruppen darzustellen?
Zusätzlich erlaubt der vorgestellte Algorithmus das modulare %> modulare? was meinst du damit?
zusammenfassen von Gruppen zu Pseudoknoten, %> ist nicht mit dem Rest konsistent: eher erlaubt das Anzeigen von Gruppen im geöffneten und geschlossenen Zustand
bzw. deren modulare Expansion, da das Ausgangslayout alle Gruppen zu solchen Pseudoknoten zusammenfasst. 
Besonderer Fokus wurde zusätzlich auf das Ästhetikkriterium gelegt, das das Layout sich durch \glqq Öffnen und Schließen\grqq\  %> haben sonst auch nirgends Anführungszeichen
% > Zur Not eher noch oben definieren, was mit Öffnen oder Schließen gemeint ist. egtl sollte es aber klar sein und wird auch iwie erklärt
der Gruppen nur gering verändert.
% Die Wahl des Layouts und der Algorithmen erlaubt es, ein einfaches Vorgehen für den Wechsel von Layouts bei einer Interaktion. 

% > Warum sollte man das sagen wollen dass es unseren Umfang überstieg? Würde eher einfach sagen, dass es nicht gemacht wurde. fertig. kein grund :D
Eine konkrete Implementierung und entsprechende Auswertung selbiger lag nicht im Umfang dieser Arbeit.
%- daher wurden diese nicht durchgeführt. 
Bei einer groben Abschätzung wurde lediglich kein %> lediglich kein? 
Faktor gefunden, der die Laufzeit üblicher kräftebasierter Verfahren dominieren %> dominieren?
könnte.
%\todo[inline]{Citation needed?}

Zukünftige Arbeiten können sich mit der in dieser Arbeit unterlassenen Implementation und Auswertung beschäftigen. 
Ein Teil einer solchen Implementierung sollte die Parameter, welche bei der Beschreibung des Algorithmus bewusste nicht festgelegt wurden, konkret abschätzen.
% Außerdem Vorschläge: Öfters hoch und runter bei Layoutberechnung. Ports nicht ganz fest. Kanten abrunden

Alternativ sind auch Ausarbeitungen und Untersuchungen der alternativen Lösungsansätze, die diese Arbeit nur grob umrissen hat, möglich.

