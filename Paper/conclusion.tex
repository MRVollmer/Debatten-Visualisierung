\chapter{Zusammenfassung und Ausblick}
In dieser Arbeit haben wir mit dem Layout-Problem für Argumentkarten beschäftigt, insbesondere bezüglich semantischen Gruppen von Argumenten, d.h. Gruppenzugehörigkeiten ein Teil des zu zeichnenden Graphen sind im Gegensatz zu Gruppen, die als Teil des Layouts bestimmt werden (wie bspw. Graph-Clustering).

Für dieses Problem haben wir drei Lösungsansätze vorgestellt und verglichen. Einen dieser drei Ansätze haben wir weiter ausgearbeitet und das Vorgehen in dieser Arbeit detailliert beschrieben:

Dieser Ansatz verwendet und erweitert bisherige kräftebasierte Algorithmen um vordefinierte Gruppen beim Layout zu respektieren. Zusätzlich erlaubt der vorgestellte Algorithmus das modulare zusammenfassen von Gruppen zu Pseudoknoten, bzw. deren modulare Expansion, da das Ausgangslayout alle Gruppen zu solchen Pseudoknoten zusammenfasst. Besonderer Fokus wurde zusätzlich auf das Ästhetik-Kriterium, dass das Layout sich durch \glqq Öffnen und Schließen\grqq\  der Gruppen nur gering verändert.

Eine konkrete Implementierung und entsprechende Auswertung selbiger übersteigen allerdings den Umfang dieser Arbeit, daher wurden diese nicht durchgeführt. Bei einer groben Abschätzung wurde lediglich kein Faktor gefunden der die Laufzeit üblicher kräftebasierter Verfahren dominieren könnte.
\todo[inline]{Citation needed?}

Zukünftige Arbeiten können sich mit der in dieser Arbeit unterlassenen Implementation und Auswertung beschäftigen. Ein Teil einer solchen Implementierung sollte die Parameter, welche bei der Beschreibung des Algorithmus bewusste nicht festgelegt wurden, konkret abschätzen.

Alternativ sind auch Ausarbeitungen und Untersuchungen der alternativen Lösungsansätze, die diese Arbeit nur grob umrissen hat, möglich.

